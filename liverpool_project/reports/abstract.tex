{
\newpage
\vspace*{\fill}
{
	\centering
	\section*{Abstract}
}
\addcontentsline{toc}{section}{Abstract}
Microcontrollers have become a ubiquitous and vital component in a wide range of applications, from embedded systems to Internet of Things (IoT) devices.
With the growing demand for efficient and versatile computing solutions, the RISC-V (Reduced Instruction Set Computing - V) Instruction Set Architecture (ISA) has emerged as
a promising alternative to proprietary and traditional ISAs.


This thesis presents the design and implementation of a RISC-V microcontroller tailored to meet the demands of modern computing systems.
The proposed microcontroller architecture leverages the inherent advantages of RISC-V, including simplicity, modularity, and extensibility,
to deliver enhanced performance and versatility compared to existing solutions.


The research begins with a comprehensive study of the RISC-V ISA, identifying its key features and design principles.
Based on this analysis, a custom microarchitecture is devised to optimize critical components such as instruction decoding, pipeline stages, and data forwarding mechanisms.
The microcontroller is then synthesized and simulated using industry-standard Electronic Design Automation (EDA) tools to assess its performance, and area utilization.


The results of this research illustrate the feasibility and benefits of employing RISC-V as the foundation for designing microcontrollers.
The proposed architecture's ability to achieve improved performance and adaptability to application-specific
needs lays the groundwork for broader adoption of RISC-V in future computing systems.


In conclusion, this thesis contributes to the growing body of knowledge surrounding RISC-V microcontroller design and serves as a stepping stone for
further advancements in the field of open-source and customizable processor architectures. As RISC-V gains momentum in the industry,
the developed microcontroller has the potential to drive innovation and address the ever-evolving demands of the digital era.
\vspace*{\fill}
}
