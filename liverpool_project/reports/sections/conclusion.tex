\chapter{Conclusion}
Building a simple microcontroller using the RISC-V architecture represents an exciting and innovative area of 
research in computer science and embedded systems. 
The RISC-V instruction set architecture (ISA) provides a flexible and open-source foundation for designing 
custom microcontrollers tailored to specific applications. 
Here is a conclusion on this work of building a simple microcontroller using RISC-V:

\paragraph*{Open-Source Advantages}
The use of RISC-V as the basis for a microcontroller offers significant advantages, notably openness and flexibility. Researchers and developers can access and modify the ISA freely, fostering innovation and enabling customized solutions for diverse applications.
\paragraph*{Scalability}
RISC-V microcontrollers can be designed to scale from simple, low-power devices to more complex systems, making them suitable for a wide range of use cases. This scalability allows researchers to explore various design choices and trade-offs.
\paragraph*{Educational Value}
Building a simple RISC-V microcontroller is an educational endeavor with practical benefits. It provides opportunities for students and enthusiasts to learn about computer architecture, digital design, and embedded systems by working with a real-world, open-source ISA.
\paragraph*{Customization}
Researchers can tailor the microcontroller's architecture and features to meet specific application requirements. This customization can lead to energy-efficient, cost-effective solutions optimized for particular tasks.
\paragraph*{Challenges and Trade-Offs}
While RISC-V offers flexibility, designing a microcontroller from scratch can be complex and time-consuming. Researchers must carefully consider trade-offs between power efficiency, performance, and resource utilization to achieve their design goals.

Over the course of this work, it can also be concluded that the designed microcontroller works both in simulation and with hardware and can be expanded at will.
There are many potential optimazations that can be done such as choosing a better architecture and building a more coincise hardware. 

\section{Further Works}
In order to fully usable as a microcontroller, there are many features that will need to be implemented. Below is a list of some potential features
\begin{itemize}
    \item More Instructions
    \item Interrupts
    \item Timers
    \item Support for procedures
    \item Multiplexing special and general I/O
\end{itemize}
