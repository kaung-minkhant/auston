\subsection{Introduction}
This section will conduct testings on the CPU without peripheral connections.
The CPU will be tested with a sample program and various memories will be tested against the expected memory contents.
The test will be carried out in ModelSim.

\subsection{Sample Test Program}
In the following program, a simple fibonacci number generator is implemented.
The program will take in the number of fibonacci number to be generated in the first instruction.
It will proceed to execute the program as described in Table \ref{program:sample_simulation}.
This demonstrates the data dependencies and data dependencies during branching instructions.
The expected memory and the simulated memory content will be compared to confirm the operation of the CPU.
\begin{table}[!h]
    \centering
    \caption{Sample Simulation Program}
    \label{program:sample_simulation}
    \begin{tabular}{|c|l|l|}
        \hline
        \textbf{Line} & \multicolumn{1}{c|}{\textbf{Instruction}} & \multicolumn{1}{c|}{\textbf{Description}}   \\ \hline
        1             & addi x1 x0 3                              & setting count = 3                           \\ \hline
        2             & addi x2 x0 1                              & Initialize temp1 as 1                       \\ \hline
        3             & addi x3 x0 1                              & Initialize temp2 as 1                       \\ \hline
        4             & addi x1 x1 -2                             & does not need to iterate the first two loop \\ \hline
        5             & L1: add x4 x2 x3                          & output = temp1 + temp2                      \\ \hline
        6             & add x2 x0 x3                              & temp1 = temp2                               \\ \hline
        7             & add x3 x0 x4                              & temp2 = output                              \\ \hline
        8             & addi x1 x1 -1                             & count--                                     \\ \hline
        9             & bne x1 x0 L1                              & loop if count is not 0                      \\ \hline
    \end{tabular}
\end{table}

\subsection{Expected and Simulated Register File Content}
Table \ref{table:expected_rf} described the expected register file after the execution of the sample program.
\begin{table}[]
    \centering
    \caption{Expected Register File for Sample Program}
    \label{table:expected_rf}
    \begin{tabular}{|c|c|}
        \hline
        \textbf{Memory Location} & \textbf{Value (Decimal)} \\ \hline
        0                        & 0                        \\ \hline
        1                        & 0                        \\ \hline
        2                        & 1                        \\ \hline
        3                        & 2                        \\ \hline
        4                        & 2                        \\ \hline
    \end{tabular}
\end{table}

\insertWaveform{simulation_testing_rf}{0.5}{Simulated Register FIle Content}{waveform:simulated_rf}

As can be observed, the simulation, the final register file contents match the expected register file contents, thus confirming the operation of the CPU.
Further testings will be contacted during hardware testings in a later section.

\newpage
\subsection{Pipeline Execution}
Figure \ref{graphic:sample_pipeline_execution} describes the pipeline execution with respect to the clock cycles and the instructions.
In the figure, the required data forwardings are indicated with a red arrow.
The arrow starts from the pipeline stage that the data comes from and ends on the pipeline stage where the data is needed.

\insertGraphic{sample_pipeline_execution}{0.3}{0}{Sample Pipeline Execution}{graphic:sample_pipeline_execution}
