\section{Input/Output (IO) Module}
I/O Module is used to communicate with the outside world.
It contains input ports, which will take in sensor signals, and output ports, which will drive other devices.
It also features communication ports, so that the microcontroller can communicate with different type of complex sensor systems.
This will enable microcontroller to talk to other systems.
Thus, this module is one of the core module of Peripherals.

This module is structured into two parts: general I/O and special I/O.
General I/O features normal inputs and outputs. They can take in digital inputs such as switches, and output digital logic signals.
Special I/O features other families of peripherals, such as communication protocols like SPI, UART, TIMERS and other modules that might be added
in the future.

In order to communicate with CPU, this module is connected to CPU via Data Memory in CPU as mentioned in CPU section.
All the ports required to communicate between CPU and peripherals is provided.


\subsection{Block Diagram and I/O}
\insertBlockDiagram{io_block_diagram}{1}{I/O Module Block Diagram}{block_diagram:io}

\begin{table}[!h]
    \centering
    \caption{Input/Output Table of I/O Module}
    \label{table:io_io}
    \resizebox{\textwidth}{!}{
        \begin{tabular}{|ll|l|l|}
            \hline
            \multicolumn{1}{|c|}{\textbf{Signal}}              & \multicolumn{1}{c|}{\textbf{Constituents}} & \multicolumn{1}{c|}{\textbf{Description}}          & \multicolumn{1}{c|}{\textbf{Port}} \\ \hline
            \multicolumn{2}{|l|}{general\_io}                  & general io port connected to outside world & Input/Output                                                                            \\ \hline
            \multicolumn{2}{|l|}{special\_io}                  & special io port connected to outside world & Input/Output                                                                            \\ \hline
            \multicolumn{1}{|l|}{\multirow{3}{*}{general\_io}} & general\_io\_dir                           & direction of General I/O ( 0 = output, 1 = input ) & Input                              \\ \cline{2-4}
            \multicolumn{1}{|l|}{}                             & general\_io\_data\_in                      & data to be output by general io                    & Input                              \\ \cline{2-4}
            \multicolumn{1}{|l|}{}                             & general\_io\_data\_out                     & data read from general io                          & Output                             \\ \hline
            \multicolumn{1}{|l|}{\multirow{5}{*}{uart1}}       & uart1\_controls                            & control bits of UART1                              & Input                              \\ \cline{2-4}
            \multicolumn{1}{|l|}{}                             & uart1\_flags                               & flag bits of UART1                                 & Output                             \\ \cline{2-4}
            \multicolumn{1}{|l|}{}                             & uart1\_data\_in                            & data to be transmitted by UART1                    & Input                              \\ \cline{2-4}
            \multicolumn{1}{|l|}{}                             & uart1\_data\_out                           & 8 bit data read by UART1                           & Output                             \\ \cline{2-4}
            \multicolumn{1}{|l|}{}                             & uart1\_data\_out\_32                       & 32 bit data read by UART1                          & Output                             \\ \hline
            \multicolumn{1}{|l|}{\multirow{4}{*}{spi1}}        & spi1\_controls                             & control bits for SPI1                              & Input                              \\ \cline{2-4}
            \multicolumn{1}{|l|}{}                             & spi1\_flags                                & flag bits for SPI1                                 & Output                             \\ \cline{2-4}
            \multicolumn{1}{|l|}{}                             & spi1\_data\_in                             & data to be transmitted by SPI1                     & Input                              \\ \cline{2-4}
            \multicolumn{1}{|l|}{}                             & spi1\_data\_out                            & data read by SPI1                                  & Output                             \\ \hline
            \multicolumn{1}{|l|}{\multirow{5}{*}{i2c1}}        & i2c1\_controls                             & control bits for I2C1                              & Input                              \\ \cline{2-4}
            \multicolumn{1}{|l|}{}                             & i2c1\_flags                                & flag bits for I2C1                                 & Output                             \\ \cline{2-4}
            \multicolumn{1}{|l|}{}                             & i2c1\_data\_in                             & data to be transmitted by I2C1                     & Input                              \\ \cline{2-4}
            \multicolumn{1}{|l|}{}                             & i2c1\_data\_out                            & data read by I2C1                                  & Output                             \\ \cline{2-4}
            \multicolumn{1}{|l|}{}                             & i2c1\_addr                                 & address to transmitted by I2C1                     & Input                              \\ \hline
            \multicolumn{2}{|l|}{clk}                          & CPU clock                                  & Input                                                                                   \\ \hline
            \multicolumn{2}{|l|}{clk\_main}                    & system clock                               & Input                                                                                   \\ \hline
            \multicolumn{2}{|l|}{reset\_n}                     & system reset                               & Input                                                                                   \\ \hline
        \end{tabular}
    }
\end{table}

\newpage
\subsection{Usage}
This module is an extensible module. If designers decided to add more functinalities that would work with the outside world,
this module will need to be modified to add new modules and related ports.
Currently one of SPI, I$^2$C, UART and general io modules are added.
Furthermore, the new module ports will need to be mapped in Data Memory of CPU as well.
\subsubsection{Initialization}
The system frequency must be initiated when using the module, since some of the connected module use that infomation to calculate their own clocks.

\newpage
\subsection{Architecture and Implementation}
Special I/O modules such as SPI and I$^2$C are brought in as portable modules in this I/O module by simply instanctiating them.
The architecture of those modules have be explored above.

As for general I/O, the heart and soul of it is iopin module, and source code can be found in \url{https://github.com/kaung-minkhant/risc-v-deca/blob/master/peripherals/io/iopin.vhd}.
This module is responsible for retriving pin direction data, the date to write, and reading data.
The read and write are synchronized to system clock.
Thus the sensor values at the input pins are sampled at system clock frequency.
If the direction is 1 (input), the pin is released from drivers, which means setting it as tristate, so that the correct data can be read. 

