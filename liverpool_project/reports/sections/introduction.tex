\chapter{Introduction}
\section{Project Background}

In recent years, the rapid advancement of technology has led to an exponential growth in the number of embedded systems and Internet of Things (IoT) devices. 
These systems demand efficient, low-power, and versatile microcontrollers capable of executing complex tasks while maintaining cost-effectiveness. 
To address these challenges, the development of a RISC-V-based microcontroller presents an attractive and innovative solution.


The Reduced Instruction Set Computing - V (RISC-V) Instruction Set Architecture (ISA) is an open standard, freely available for anyone to use, modify, and implement. 
Its modular and scalable design has gained significant attention within the semiconductor industry and academic community. 
Unlike proprietary ISAs, RISC-V fosters collaboration and fosters a rich ecosystem of open-source hardware and software projects, 
leading to reduced development costs and increased accessibility.


This project aims to design a RISC-V microcontroller that leverages the benefits of the RISC-V ISA, 
tailored to meet the specific requirements of modern computing applications. The motivation behind this work aims to provide the detail designs of several parts of 
the microcontroller, aiming to act a reference and a base for future design and implementation of the microcontroller core using RISC-V ISA.


\section{Aims and Objectives}
This research is proposed for construction of RISC-V Hardware using VHDL and implemented on FPGA.
The objectives of this research are as follow.
\begin{enumerate}
	\item To build a complete microcontroller, yet simple enough to follow, based on RISC-V RV32E ISA using VHDL.
	\item To offer detail documentation of the implementation of the hardware
	\item To offer hardware extension for future developement
	\item To offer assembler for firmware creation
	\item To offer implementation support for hardware designers to build their own hardwares based on the required projects
\end{enumerate}
\section{Expected Outcomes}
The expected outcomes of the research are as follows
\begin{enumerate}
	\item The CPU must work based on the specified ISA, in this case, RICS-V RV32E ISA
	\item The CPU must be tested
	\item The overall micro-controller must have standard peripherals and they must be tested
	\item The documentation of all of the hardware implements must be provided
	\item The research will be published on GitHub Repo

\end{enumerate}

\section{Report Structure}
This report has been broken down into sections.
\begin{enumerate}
  \item Chapter 1: Introduction - Introduction above this work
  \item Chapter 2: Literature Review - A review of current microcontroller works around RISC-V
  \item Chapter 3: Microcontroller CPU Design and Implementation - RISC-V CPU Core Design and Implementations
  \item Chapter 4: Peripheral Designs and Implementation - Design of peripherals for CPU Core
  \item Chapter 5: Simulation Testings - Testings of CPU and Peripherals in simulation
  \item Chapter 6: Conclusion - Conclusion that can be drawn from this work
\end{enumerate}

\section{Summary}
By successfully designing and implementing a RISC-V microcontroller with a focus on performance and versatility, 
this project seeks to make a valuable contribution to the field of embedded systems and IoT devices. 
Embracing the open-source philosophy, the project's outcomes will be shared with the broader community, 
encouraging collaboration and pushing the boundaries of what RISC-V-based microcontrollers can achieve.
