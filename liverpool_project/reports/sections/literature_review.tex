\chapter{Literature Review}
The development of a RISC-V microcontroller with a focus on simplicity and comprehensive documentation availability has gained significant interest in both 
academic and industrial circles. RISC-V's open-source nature allows for greater accessibility and collaboration, making it an attractive choice for microcontroller designs. 
This literature review aims to explore relevant research and publications related to RISC-V-based microcontroller designs, 
with an emphasis on simplicity in architecture and the availability of extensive documentation.

Researchers can tailor the microcontroller's architecture and features to meet specific application requirements. 
This customization can lead to energy-efficient, cost-effective solutions optimized for particular tasks\cite{riscvISA}.

``The RISC-V Instruction Set Manual, Volume I: User-Level ISA" by Andrew Waterman provides an in-depth documentation of the RISC-V user-level instruction set architecture, 
offering a comprehensive understanding of the ISA's base instructions. It serves as a crucial reference for microcontroller designers seeking to implement 
a simple and standard-compliant RISC-V core. The clear and well-structured presentation of the ISA facilitates an uncomplicated design process, 
encouraging developers to create efficient and robust microcontrollers \cite{riscvISA}.


The paper by Daniel Brisk, focuses on architectural simplicity while maintaining high performance and low power consumption. 
RI5CY's generator includes customization options, allowing designers to tailor the microcontroller for specific applications. 
The work underscores the importance of simplicity in the microcontroller design process while enabling adaptability to 
diverse use cases \cite{ri5cy}.


Rocket Chip, RISC-V Processer by Yunsup Lee, provides design philosophy emphasizes simplicity, enabling ease of understanding, modification, and extension. 
Rocket Chip is highly configurable, allowing designers to create microcontrollers suited to their specific requirements. 
The paper highlights the significance of open-source projects in facilitating collaboration and fostering a community-driven approach to 
microcontroller design \cite{rocket_chip}.


Frank K. Gürkaynak present PULPino, an open-source RISC-V processor designed for Internet of Things (IoT) applications. The focus of the project is on 
simplicity and energy efficiency, making it suitable for resource-constrained devices. PULPino's architecture is thoroughly documented, 
promoting ease of use and customization for specific IoT use cases \cite{pulpino}.


Ibex, by Stefan Wallentowitz, is an open-source, parameterizable RISC-V processor core designed for flexibility and simplicity. 
The paper highlights the core's configurability and comprehensible architecture, which allows designers to optimize it for various applications. 
Ibex's extensive documentation and open-source nature facilitate 
community-driven improvements and encourage knowledge exchange \cite{ibex}.


VexRiscv by Charles Papon is an open-source RISC-V CPU implementation in SpinalHDL. The project emphasizes clarity and modularity, 
allowing developers to easily understand and modify the processor core. The repository includes extensive documentation, 
making it a valuable resource for those interested in building customizable RISC-V microcontrollers \cite{vexriscv}.


The paper for AnnikaCore, by Yunrui Zhang, discusses the growing market demand for embedded IoT processors due to the booming IoT industry. 
It highlights the RISC-V instruction set architecture, known for its concise coding and modular extensions, 
as an ideal choice for embedded IoT processors. The paper presents the design of a 3-stage pipelined scalar micro-out-of-order 
processor based on RISC-V's RV32IMA instruction set. The processor has been verified through simulation and FPGA prototype, 
showing functional correctness with a Coremark performance of 2.93 Coremark/MHz. 
The final implementation used SMIC 180nm process with a main frequency of 50MHz, resulting in a core circuit of 35K gate 
and a power consumption of 0.20 mW/MHz \cite{annikacore}.

The article authored by Aaron Elson Phangestu explores the development of a soft processor core with a five-stage pipeline. 
This core is designed based on the RISC-V RV32I Base Integer Instruction Set Architecture, which is an openly available and 
modifiable standard ISA. The process of creating this processor core involved adhering to FPGA design principles. 
Initially, they translated the design specifications into VHDL Hardware Description Language and subsequently conducted 
simulations using the ModelSim simulation environment. After verification, the core underwent an analysis of 
resource utilization, critical path assessment, and maximum operating frequency. 
Finally, the processor core was programmed onto a physical Cyclone IV EP4CE6E22C FPGA. \cite{32bitcore}

The above works, although rich in information and advances in RISC-V Microcontroller technology, lack in a way that prevents a beginner to start designing a customized microcontroller along with providing a comprehensive documentation. 
They lack either in the detail design steps or in the implementation itself using a hardware description language.
This work forcus around designing a simple RISC-V microcontroller with a simple enough documentation so that a beginner will have an entry to start designing the desired 
microcontroller. This work also provides a way for producing a customized assembler for producing the machine code for the controller.
