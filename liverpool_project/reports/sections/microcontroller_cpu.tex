\chapter{Microcontroller CPU Design and Implementation}

\section{Introduction}
CPU design is the process of creating the central processing unit, the core component of a computer responsible for executing instructions and calculations.
It comprises components like the Control Unit, Arithmetic Logic Unit, and registers, following an Instruction Set Architecture (ISA) that defines the CPU's capabilities.
CPUs may often employ pipeline architectures, cache memory, and parallel processing to enhance performance.
In this work, a simple CPU Architecture is devised and designed.

\section{System Design}
\insertBlockDiagram{CPU Block Diagram Overview}{0.5}{Overview of CPU}{block_diagram:cpu_overview}
Figure \ref{block_diagram:cpu_overview} describe the block diagram overview for the designed CPU. This diagram seves as a general overview without the complications
of pipelining. In this diagram, it can be observed that Havard CPU architecture is used, since Instruction Memory and Data Memory are seperated.
It can also be seen that the flow of data between components, which will be explained below.


At start up, instruction fetch operation is performed on Instruction Memory, and at the same time
increment the program counter by 4. Then, the fetched instruction is decoded and fed into control unit and immediate generator.
The control unit asserts and clear the control lines necessary for the operation of CPU.
Immediate generator is resposible for generating corrent immediates required for the operation of ALU.
Then data in the register file is accessed by the bit fields of the instructions. These data and the immediate
is used for ALU operations. If branch instructions are used, the program counter is set with the appropriate address value.
Finally, datas are stored into Data Memory if store instructions are used. The summerized flow diagram can be see in
Figure \ref{graphic:flow_diagram}.

% TODO: FIX THIS DIAGRAM IF HAVE TIME
\insertGraphic{CPU Flow Diagram}{0.8}{0}{Flow Diagram of CPU}{graphic:flow_diagram}

\section{Instruction Formats}
In order to support pipelining and modularity, only certain types of instruction formats are
supported in RISC-V architectures. The instruction types supported in this work are R-type, I-type, SB-type and S-types.
The instruction formats can be seen in Figure \ref{graphic:instruction_formats}. It should be noted that several
instruction fields occupy the same location in the instruction format, which simplify hardware implementation and decoding
the instructions.

\insertGraphic{instruction_formats}{0.4}{0}{Instruction Formats}{graphic:instruction_formats}

Serveral bit fields highlight in Figure \ref{graphic:instruction_formats} are explained in Table \ref{table:instruction_fields}

\begin{table}[!h]
    \centering
    \caption{Instruction Fields Definition}
    \label{table:instruction_fields}
    \resizebox{0.8\textwidth}{!}{
        \begin{tabular}{|l|l|}
            \hline
            \textbf{Field Name} & \textbf{Definition}                              \\ \hline
            opcode              & instruction code                                 \\ \hline
            rd                  & destination register in Register File            \\ \hline
            rs1                 & source 1 register in Register File               \\ \hline
            rs2                 & source 2 register in Register File               \\ \hline
            immediate/immed     & immediate to be used for ALU operation           \\ \hline
            funct3              & 3 bits function code to distinguish instructions \\ \hline
            funct7              & 7 bits function code to distinguish instructions \\ \hline
        \end{tabular}
    }
    
\end{table}
\section{Support Instructions}
The instructions supported in this work will be outlined here along with the instruction type they belong to.
These instructions are enough to make simple computations, though not complex ones. There are three types all of these
supported instructions fall into: Arithmetic, Load/Store, Branching Instructions.

\subsection{Arithmetic Instrucitons}
Arithmetic Instructions can be broken down into two sub types, based on their usage:
Register Arithmatic and Constant Arithmetic. All arithmetic instructions must be performed
using registers and constants. If the data to be used is in the Data Memory, load/store instructions
must be utilized in order to load in the required data into a register.

\paragraph*{Register Arithmetic Instructions}
These types of instructions utilize two pieces of data stored in registers to perform arithmetic
operations on them. Then the result is stored back into the register specified.

\begin{table}[!h]
    \centering
    \caption{Register Arithmetic Instructions}
    \label{table:register_arithmatic_instructions}
    \resizebox{\textwidth}{!}{
        \begin{tabular}{|l|l|l|c|}
            \hline
            \multicolumn{1}{|c|}{\textbf{Instruction}} & \multicolumn{1}{c}{\textbf{Description}}                      & \multicolumn{1}{|c|}{\textbf{Usage}} & \textbf{Instruction Type} \\ \hline
            add                                        & \makecell{add two registers, rs1 and rs2,\\ then store the result into rd} & add rd rs1 rs2                       & R                         \\ \hline
            sub                                        & \makecell{subtract rs2 from rs1, \\then store the result into rd }         & sub rd rs1 rs2                       & R                         \\ \hline
            or                                         & \makecell{binary OR between rs1 and rs2, \\then store the result into rd}  & or rd rs1 rs2                        & R                         \\ \hline
            and                                        & \makecell{binary AND between rs1 and rs2, \\then store the result into rd} & and rd rs1 rs2                       & R                         \\ \hline
        \end{tabular}
    }
\end{table}

\paragraph*{Constant Arithmetic Instructions}
Since arithmetic operations with constants are very popular in computations, instructions with
constant fields are provided. In these instructions, the constants are stored as signed 12 bit integer.
Thus, they operate the data from one register and the constant, then store the result back into
specified destination register.
\begin{table}[!h]
    \centering
    \caption{Constant/Immediate Arithmetic Instructions}
    \label{table:constant_arithmetic_instructions}
    \resizebox{\textwidth}{!}{
        \begin{tabular}{|l|l|l|c|}
            \hline
            \multicolumn{1}{|c|}{\textbf{Instruction}} & \multicolumn{1}{c}{\textbf{Description}}                                  & \multicolumn{1}{|c|}{\textbf{Usage}} & \textbf{Instruction Type} \\ \hline
            addi                                       & \makecell{add rs1 and signed constant, \\then store the result into rd }               & addi rd rs1 constant                 & I                         \\ \hline
            ori                                        & \makecell{binary OR between rs1 and signed constant, \\then store the result into rd}  & ori rd rs1 constant                  & I                         \\ \hline
            andi                                       & \makecell{binary AND between rs1 and signed constant, \\then store the result into rd} & andi rd rs1 constant                 & I                         \\ \hline
        \end{tabular}
    }
\end{table}

\newpage
\subsection{Load/Store Instructions}
As the name suggests, these instructions fall into two categories: Load and Store Instructions.
These instructions are mainly used to access and interact with the Data Memory in the system.
Since they interact with Data Memory, a way to specify the address of Data Memory must be provided.
It is accomplished by using offseting address method. The base address is stored in rs1 while offset constant
is provided via immediate in the instruction. This way, all the addresses will be covered without the need to
use a very large immediate, which will require more bits. As can be seen in the instruction format in
Figure \ref{graphic:instruction_formats}, the immediate fields takes up only 12 bit of data, thus, offset
addressing will enable full coverage of Data Memory.

\begin{table}[!h]
    \centering
    \caption{Load/Store Instructions}
    \label{table:load/store_instructions}
    \resizebox{\textwidth}{!}{
        \begin{tabular}{|l|l|l|c|}
            \hline
            \multicolumn{1}{|c|}{\textbf{Instruction}} & \multicolumn{1}{c}{\textbf{Description}}                              & \multicolumn{1}{|c|}{\textbf{Usage}} & \textbf{Type} \\ \hline
            ld                                         & \makecell{load 32 bit data from data memory \\at location of rs1+offset into rd}  & ld rd rs1 offset                     & I             \\ \hline
            sd                                         & \makecell{store 32 bit data from rs2 into data memory \\at location of rs1+offset} & sd rs2 rs1 offset                    & S             \\ \hline
        \end{tabular}
    }
\end{table}

\subsection{Branching Instruction}
Branching instructions are required in order to do decision making within the CPU.
Notably loops and conditionals. Traditionally, two types of branching instruction are required, namely
conditional branching and unconditional branching. In this work, only conditional branching
has been implemented. If a programmer requires unconditional branching, condtional branching with a condition of always true
can be utilized. The destination instruction address is calculated using the current Program Counter, which is the current instruction address,
and the offset. This addressing is called PC-relative addressing. Since the source code can be several lines long,
using this kind of addressing solves the problem of storing large addresses in the offset/immediate fields in the instruction.

When calculating offset for branching, the memory alignment will need to be kept in mind.
Thus if current instruction wants to jump to previous instruction, that would be the offset of -4.
However, as per RISC-V standards, this value is halved, since there is a shift-left-once hardware inside the branch calculator.
Therefore, the actual offset will be -2.

\begin{table}[!h]
    \centering
    \caption{Branching Instructions}
    \label{table:branching_instructions}
    \resizebox{\textwidth}{!}{
        \begin{tabular}{|l|l|l|c|}
            \hline
            \multicolumn{1}{|c|}{\textbf{Instruction}} & \multicolumn{1}{c}{\textbf{Description}}                                    & \multicolumn{1}{|c|}{\textbf{Usage}} & \textbf{Type} \\ \hline
            beq                                        & \makecell{branch to the instruction address of \\pc+offset if rs1 and rs2 are equal}     & beq rs1 rs2 offset                   & SB            \\ \hline
            bne                                        & \makecell{branch to the instruction address of \\pc+offset if rs1 and rs2 are not equal} & bne rs1 rs2 offset                   & SB            \\ \hline
        \end{tabular}
    }
\end{table}

\subsection{Opcodes}
Below lists the opcodes assigned for each instruction type. Note that there are two subtypes within I-type instructions.
\begin{table}[!h]
    \centering
    \caption{Opcodes for Each Instruction Type}
    \label{table:opcodes}
    \begin{tabular}{|c|l|}
        \hline
        \textbf{Instruction Type} & \multicolumn{1}{c|}{\textbf{Opcode}} \\ \hline
        R                         & 0110011                              \\ \hline
        I - Arithmatic            & 0010011                              \\ \hline
        I - Load                  & 0000011                              \\ \hline
        S                         & 0100011                              \\ \hline
        SB                        & 1100011                              \\ \hline
    \end{tabular}
\end{table}

\section{Program Counter}
Program Counter module is responsible for keeping track of where the current instruction being executed is.
This module accomplish this task by remembering the address of instruction in the Instruction Memory
This address is advanced by 4, if the next instruction from the Instruction Memory is required.
When branching, the desired address of the instruction in the Instruction Memory is set in the
Program Counter as well, thus forming the integral part of the CPU. The operation such as
stalling is performed here as well, by stopping the Program Counter from advancing. The reason of advancing the
Program Counter will be discussed in Instruction Memory Section.

\subsection{Implementation and Operation}
The source code for Program Counter
can be accessed using \url{https://github.com/kaung-minkhant/risc-v-deca/blob/master/cpu/pc_container.vhd}.
The overall block diagram can be seen in Figure \ref{block_diagram:pc}
\insertBlockDiagram{pc_block_diagram}{1}{Block Diagram of Program Counter}{block_diagram:pc}

\begin{table}[!h]
    \centering
    \caption{Input/Output of Program Counter}
    \label{table:io_pc}
    \resizebox{0.7\textwidth}{!}{
        \begin{tabular}{|l|l|l|}
            \hline
            \multicolumn{1}{|c|}{\textbf{Signal}} & \multicolumn{1}{c|}{\textbf{Description}} & \multicolumn{1}{c|}{\textbf{Port}} \\ \hline
            branch\_target                        & the address of instruction to branch to   & Input                              \\ \hline
            stall                                 & stalling signal                           & Input                              \\ \hline
            branch\_condition                     & indicates if it is branch instruction     & Input                              \\ \hline
            controls                              & branch and load                           & Input                              \\ \hline
            reset\_n                              & system reset                              & Input                              \\ \hline
            clk                                   & CPU clock                                 & Input                              \\ \hline
            clk\_main                             & system clock                              & Input                              \\ \hline
            pc\_address                           & Address of current instruction            & Output                             \\ \hline
        \end{tabular}
    }
\end{table}


\paragraph*{Operation}
By default, with every rising edge of the CPU Clock, the PC is advanced by 4. However, if \io{branch\_condition}
is asserted, the pc will take \io{branch\_target} into its internal buffer, but not set it as output.
Only when the \sig{branch} signal within \io{controls} is asserted, the \io{branch\_target} is
asserted as output on \io{pc\_address}. Otherwise, the output \io{pc\_address} will be the pc which is
advanced by 4. When \io{stall} is asserted, the Program Counter is `freezed' at current instruction address.

\section{Instruction Memory}
The Instruction Memory is responsible for storing the instructions to be executed.
In order to simplify the instruction decoding and further extenstion of the cpu, the byte addressing is used.
In byte addressing, the address incremented by each byte.
In the 32 bit system, it means 32 bit instructions are seperated by 4 bytes.
If the first instruction is located at address 0, the next instruction will be at address 4.
Thus the Program Counter is advanced by 4. This is called memory alignment, in this case, it is aligned with 32 bit.

\subsection{Implementation and Operation}
The source code for Program Counter
can be accessed using \url{https://github.com/kaung-minkhant/risc-v-deca/blob/master/cpu/instruction_memory.vhd}.
The overall block diagram can be seen in Figure \ref{block_diagram:im}
\insertBlockDiagram{im_block_diagram}{1}{Block Diagram of Instruction Memory}{block_diagram:im}

\begin{table}[!h]
    \centering
    \caption{Input/Output of Instruction Memory}
    \label{table:io_im}
    \resizebox{0.7\textwidth}{!}{
        \begin{tabular}{|l|l|l|}
            \hline
            \multicolumn{1}{|c|}{\textbf{Signal}} & \multicolumn{1}{c|}{\textbf{Description}} & \multicolumn{1}{c|}{\textbf{Port}} \\ \hline
            instruction\_address                  & address of the instruction                & Input                              \\ \hline
            instruction                           & instruction at current address            & Output                             \\ \hline
            clk                                   & CPU clock                                 & Input                              \\ \hline
            reset\_n                              & system reset                              & Input                              \\ \hline
        \end{tabular}
    }
\end{table}

\paragraph*{Operation}
The Instruction Memory only operate as a ROM module.
The instruction to be executed will be loaded in the Instruction Memory when compiling the CPU.
The instruction at the current PC on \io{instruction\_address} will be available on \io{instruction}

\section{Register File}
The Register File is responsible for storing all the necessary data required by the CPU.
As stated in the supported instructions, all arithmetic operations will be done on the data stored within the Register File.
As per the RISC-V specification, there are 32 registers. While noting these registers, the prefix `x' must be used.
For example, register 1 will be x1. Register 0, x0, is tied to 0, due to the popular usage of the constant 0.
Since most of the instructions have two source register as can be seen in Figure \ref{graphic:instruction_formats}, two address ports and two output data ports are required.
One address port and input data port is needed for destination register. The Register File can be controlled whether to write based on the control signals.

\subsection{Implementation and Operation}
The source code for Register FIle
can be accessed using \url{https://github.com/kaung-minkhant/risc-v-deca/blob/master/cpu/register_file.vhd}.
The overall block diagram can be seen in Figure \ref{block_diagram:rf}
\insertBlockDiagram{rf_block_diagram}{0.8}{Block Diagram of Register File}{block_diagram:rf}

\begin{table}[!h]
    \centering
    \caption{Input/Output of Register File}
    \label{table:io_rf}
    \resizebox{0.7\textwidth}{!}{
        \begin{tabular}{|l|l|l|}
            \hline
            \multicolumn{1}{|c|}{\textbf{Signal}} & \multicolumn{1}{c|}{\textbf{Description}} & \multicolumn{1}{c|}{\textbf{Port}} \\ \hline
            rs1                                   & source register 1                         & Input                              \\ \hline
            rs2                                   & source register 2                         & Input                              \\ \hline
            rd                                    & destination register                      & Input                              \\ \hline
            rd\_data                              & data to write for destination register    & Input                              \\ \hline
            rw                                    & write control                             & Input                              \\ \hline
            rs1\_data                             & source register 1 data                    & Output                             \\ \hline
            rs2\_data                             & source register 2 data                    & Ouput                              \\ \hline
            clk                                   & CPU clock                                 & Input                              \\ \hline
            reset\_n                              & system reset                              & Input                              \\ \hline
        \end{tabular}
    }
\end{table}

\paragraph*{Operation}
The register operates on the principle called write-before-read, which write the data first before attempting to read if the source and destination register are the same.
\io{rs1} and \io{rs2} receive the addresses of the source registers from the instruction, while \io{rd} receives the address of the destination register.
The file itseft is constructed as an array within the Register File. The read portion of the Register File is not synchronized.
The read data is available at all time regardless of the clock and control on \io{rs1\_data} and \io{rs2\_data}.
The write, however, is synchronized to the falling edge of CPU clock.
When \io{rw} is asserted, on the falling edge of the CPU clock, the data on \io{rd\_data} is written into the specified address on \io{rd}.

Normally, the result from the ALU is stored into Register File.
In order to support Load Instruction, which load the data from Data Memory into Register File, a multiplexer is implemented on \io{rd\_data}.
It is placed outside of the Register File and can be controlled by Control Unit.

\section{Immediate Generator}
Immediate Generator is responsible for extracting proper immediate from the instructions to be used as constants and offsets.
Since all data need to be aligned to 32 bit, sign extension is performed on the extracted data.
Based on each instruction, immediate fields from different parts of the instruction is extracted.

\subsection{Implementation and Operation}
The source code for Immediate Generator
can be accessed using \url{https://github.com/kaung-minkhant/risc-v-deca/blob/master/cpu/immediate_generator.vhd}.
The overall block diagram can be seen in Figure \ref{block_diagram:ig}
\insertBlockDiagram{ig_block_diagram}{1}{Block Diagram of Immediate Generator}{block_diagram:ig}

\begin{table}[!h]
    \centering
    \caption{Input/Output of Immediate Generator}
    \label{table:io_ig}
    \resizebox{0.6\textwidth}{!}{
        \begin{tabular}{|l|l|l|}
            \hline
            \multicolumn{1}{|c|}{\textbf{Signal}} & \multicolumn{1}{c|}{\textbf{Description}} & \multicolumn{1}{c|}{\textbf{Port}} \\ \hline
            Instruction                           & Instructioon                              & Input                              \\ \hline
            Immediate                             & Immediate/Constant/Offset                 & Output                             \\ \hline
        \end{tabular}
    }
\end{table}

\paragraph*{Operation}
Since the instruction formats are limited, the extraction of immediate fields from \io{Instruction} are not complicated.
Depending on opcode, funct3 and funct3 fields, the immediate fields are extracted and sign extended and set the output on \io{Immediate}.

\section{Arithmetic Logic Unit}
In a CPU, Arithmetic Logic Unit is the heart of the system.
It handles all of the arithmetic operations that is to be done on the data.
Multiple operations are supported and all of these operation are done in signed integer notation.
The following operations are supported: binary and, binary or, add and substract.

\subsection{Implementation and Operation}
The source code for Arithmetic Logic Unit
can be accessed using \url{https://github.com/kaung-minkhant/risc-v-deca/blob/master/cpu/alu.vhd}.
The overall block diagram can be seen in Figure \ref{block_diagram:alu}
\insertBlockDiagram{alu_block_diagram}{1}{Block Diagram of Arithmetic Logic Unit}{block_diagram:alu}

\begin{table}[!h]
    \centering
    \caption{Input/Output of Arithmetic Logic Unit}
    \label{table:io_alu}
    \resizebox{0.5\textwidth}{!}{
        \begin{tabular}{|l|l|l|}
            \hline
            \multicolumn{1}{|c|}{\textbf{Signal}} & \multicolumn{1}{c|}{\textbf{Description}} & \multicolumn{1}{c|}{\textbf{Port}} \\ \hline
            A                                     & Input A                                   & Input                              \\ \hline
            B                                     & Input B                                   & Input                              \\ \hline
            operation                             & Operation of ALU                          & Input                              \\ \hline
            result                                & ALU result                                & Output                             \\ \hline
            clk                                   & CPU clock                                 & Input                              \\ \hline
            reset\_n                              & system reset                              & Input                              \\ \hline
        \end{tabular}
    }
\end{table}

\paragraph*{Operation}
The 32 bit ALU can be broken down into a single 1 bit ALU.
This 1 bit ALU performs the specified ALU Operation on the given data of \io{A} and \io{B}.
The carry bits are carry away into the next 1 bit ALU in order to fully build the 32 bit ALU.
Within this 1 bit ALU, the results are set to the output based on the operation code, \io{operation}, set by the Control Unit.
The list of operation codes support in this ALU can be seen in Table \ref{table:alu_operation}. All of the results
from all 32 1-bit ALUs are aggregated into the final result of the ALU, which is \io{result}.
\begin{table}[!h]
    \centering
    \caption{Support ALU Operation and Their Operation Codes}
    \label{table:alu_operation}
    \resizebox{0.4\textwidth}{!}{
        \begin{tabular}{|l|l|}
            \hline
            \multicolumn{1}{|c|}{\textbf{Operation}} & \multicolumn{1}{c|}{\textbf{ALU Operation}} \\ \hline
            00000                                    & and                                         \\ \hline
            00001                                    & or                                          \\ \hline
            00010                                    & add                                         \\ \hline
            01010                                    & sub                                         \\ \hline
        \end{tabular}
    }
\end{table}

To perform operations with immediates, a multiplexer is inserted on the \io{B} which can be controlled from the Control Unit.
This multiplex is placed outside of the ALU module.


\section{Data Memory}
Data Memory is necessary for storing user data during program execution.
Programmer can use all registers from Register File, however, in order to implement data structures like arrays, Data Memory with a large memory space is required.
Data Memory is advantageous when storing many different data.
Programmer should only use registers to store data required for current computation.
Other results that may or may not be used on different parts of the program should be stored in Data Memory.
Data Memory is also required to create interfaces for Peripherals.
Some addresses of Data Memory can be reserved to create communtion between CPU and Peripherals

\subsection{Implementation and Operation}
The source code for Data Memory
can be accessed using \url{https://github.com/kaung-minkhant/risc-v-deca/blob/master/cpu/data_memory.vhd}.
The overall block diagram can be seen in Figure \ref{block_diagram:dm}
\insertBlockDiagram{dm_block_diagram}{0.8}{Block Diagram of Data Memory}{block_diagram:dm}

\begin{table}[!h]
    \centering
    \caption{Input/Output of Data Memory}
    \label{table:io_dm}
    \resizebox{\textwidth}{!}{
        \begin{tabular}{|l|l|l|l|}
            \hline
            \multicolumn{1}{|c|}{\textbf{Signal}} & \multicolumn{1}{c|}{\textbf{Description}}                       & \multicolumn{1}{c|}{\textbf{Constituants}} & \multicolumn{1}{c|}{\textbf{Port}} \\ \hline
            address                               & address of data in Data Memory to be written or read            &                                            & Input                              \\ \hline
            write\_data                           & data to be written                                              &                                            & Input                              \\ \hline
            write                                 & write signal                                                    &                                            & Input                              \\ \hline
            read                                  & read signal                                                     &                                            & Input                              \\ \hline
            read\_data                            & data read from the Data Memory                                  &                                            & Output                             \\ \hline
            \multirow{3}{*}{general\_pin}         & \multirow{3}{*}{the ports required for interfacing General I/O} & general\_pin\_dir                          & Output                             \\ \cline{3-4}
                                                  &                                                                 & general\_pin\_write                        & Output                             \\ \cline{3-4}
                                                  &                                                                 & general\_pin\_read                         & Input                              \\ \hline
            \multirow{5}{*}{uart1}                & \multirow{5}{*}{the ports required for interfacing UART 1}      & uart1\_flags                               & Input                              \\ \cline{3-4}
                                                  &                                                                 & uart1\_controls                            & Output                             \\ \cline{3-4}
                                                  &                                                                 & uart1\_data\_write                         & Output                             \\ \cline{3-4}
                                                  &                                                                 & uart1\_data\_read                          & Input                              \\ \cline{3-4}
                                                  &                                                                 & uart1\_data\_32\_read                      & input                              \\ \hline
            \multirow{4}{*}{spi1}                 & \multirow{4}{*}{the ports required for interfacing SPI 1}       & spi1\_flags                                & Input                              \\ \cline{3-4}
                                                  &                                                                 & spi1\_controls                             & Output                             \\ \cline{3-4}
                                                  &                                                                 & spi1\_data\_write                          & Output                             \\ \cline{3-4}
                                                  &                                                                 & spi1\_data\_read                           & Input                              \\ \hline
            \multirow{5}{*}{i2c1}                 & \multirow{5}{*}{the ports required for interfacing I2C 1}       & i2c1\_flags                                & Input                              \\ \cline{3-4}
                                                  &                                                                 & i2c1\_controls                             & Output                             \\ \cline{3-4}
                                                  &                                                                 & i2c1\_data\_read                           & Input                              \\ \cline{3-4}
                                                  &                                                                 & i2c1\_data\_write                          & Output                             \\ \cline{3-4}
                                                  &                                                                 & i2c1\_addr\_write                          & Output                             \\ \hline
        \end{tabular}
    }
\end{table}
\paragraph*{Operation}
In order to avoid some stalling problems when performing pipelining, the behavior of Data Memory is varied from the traditional Data Memory.
The normal behavior of Data Memory is to write and read data only on rising or falling edge of the clock, simplifying the hardware.
However, this approach suffer performance when performing pipelining. This will be explored in Pipeline section.



In this work, Data Memory is implemented to write on rising edge and read on the falling edge of the clock.
However, they are being controlled by two other signals, \io{write} and \io{read}.
When \io{write} is asserted on the rising edge of the clock, the data on \io{write\_data} at the given address on \io{address}.
When \io{read} is asserted on the falling edge of the clock, the data at the given address \io{address} is read, and put on \io{read\_data}.
Otherwise, the previously read data is retained.
Interfaces for connecting to Peripherals have been provided and they are stored and set output to different parts of the Data Memory.
Data Memory Mapping of the peripherals is provided when discussing peripheral connection with Data Memory.

\section{Branching Hardware}
In order to properly branch and to support different types of branching, this Branching Hardware is provided.
The responsibility of this Branching Hardware is to perform required comparisims and make desicions.

\subsection{Implementation and Operation}
The source code for Branching Hardware
can be accessed using \url{https://github.com/kaung-minkhant/risc-v-deca/blob/master/cpu/branch_controller.vhd}.
The overall block diagram can be seen in Figure \ref{block_diagram:bh}
\insertBlockDiagram{bh_block_diagram}{0.8}{Block Diagram of Branching Hardware}{block_diagram:bh}

\begin{table}[!h]
    \centering
    \caption{Input/Output of Branching Hardware}
    \label{table:io_bh}
    \resizebox{0.8\textwidth}{!}{
        \begin{tabular}{|l|l|l|}
            \hline
            \multicolumn{1}{|c|}{\textbf{Signal}} & \multicolumn{1}{c|}{\textbf{Description}}     & \multicolumn{1}{c|}{\textbf{Port}} \\ \hline
            funct3                                & 3 Bit Function field from instruction         & Input                              \\ \hline
            branch\_condition                     & branch signal indicating a branch instruction & Input                              \\ \hline
            opcode                                & Opcode field from instruction                 & Input                              \\ \hline
            rs1\_data                             & Data from rs1 register                        & Input                              \\ \hline
            rs2\_data                             & Data from rs2 register                        & Input                              \\ \hline
            branch                                & Branch signal indicating to branch            & Output                             \\ \hline
            clk                                   & CPU clock                                     & Input                              \\ \hline
            clk\_main                             & system clock                                  & Input                              \\ \hline
            reset\_n                              & system reset                                  & Input                              \\ \hline
        \end{tabular}
    }
\end{table}

\paragraph*{Operation}
This Branching Hardware takes in the funct3 field from instruction on \io{funct3} to determine which comparisim to make.
The data on \io{rs1\_data} and \io{rs2\_data} are used to make decisions from branches.
In this work, two branching comparisims are support, equal and not equal comparisims.
Once the desicion is made, the result is put on \io{branch}.
The hardware asserts \io{branch} to make a branch and clears \io{branch} to not make a branch.
This Branch Hardware is also synchronized with the clock to provide branching signal at the right instant.

\section{Control Unit}
The Control Unit is the brain of the CPU.
The responsibility of Control Unit is instruction decoding and setting necessary control signals to perform various tasks within the system.

\subsection{Implementation and Operation}
The source code for Control Unit
can be accessed using \url{https://github.com/kaung-minkhant/risc-v-deca/blob/master/cpu/control_unit.vhd}.
The overall block diagram can be seen in Figure \ref{block_diagram:cu}
\insertBlockDiagram{cu_block_diagram}{1}{Block Diagram of Control Unit}{block_diagram:cu}

\begin{table}[!h]
    \centering
    \caption{ Input/Output of Control Unit}
    \label{table:io_cu}
    \resizebox{0.7\textwidth}{!}{
        \begin{tabular}{|l|l|l|}
            \hline
            \multicolumn{1}{|c|}{\textbf{Signal}} & \multicolumn{1}{c|}{\textbf{Description}}    & \multicolumn{1}{c|}{\textbf{Port}} \\ \hline
            opcode                                & Opcode field from instructions               & Input                              \\ \hline
            stall                                 & Stall signal for stalling                    & Input                              \\ \hline
            control\_string                       & controls signals required to control the CPU & Output                             \\ \hline
        \end{tabular}
    }
\end{table}


\paragraph*{Operation}
This Control Unit is not synchronized to clk, since the PC which provided the address for the instruction is already synchronized.
This simplifies the hardware to be implemented.
The instruction decoding is done based on the opcode field of the instruction on \io{opcode}.
The decoded instruction will set the required controls on \io{control\_string}.
If \io{stall} is asserted, the \io{control\_string} is set to zero, to stall the CPU.
The value that \io{control\_string} was assigned based on instrutions can be seen in Table \ref{table:control_strings}.
The descriptions of the control signal are described in Table \ref{table:control_description}
\begin{table}[!h]
    \centering
    \caption{Controls Strings from Control Unit}
    \label{table:control_strings}
    \resizebox{\textwidth}{!}{
        \begin{tabular}{|c|c|c|c|c|c|c|c|c|}
            \hline
            \textbf{\begin{tabular}[c]{@{}c@{}}Instruction\\ Type\end{tabular}} & \textbf{pc\_controls} & \textbf{\begin{tabular}[c]{@{}c@{}}register\_file\_\\ rw\end{tabular}} & \textbf{alu\_src} & \textbf{alu\_op} & \textbf{memtoreg} & \textbf{write} & \textbf{read} & \textbf{branch\_condition} \\ \hline
            R                                   & 1                     & 1                                   & 0                 & 00               & 0                 & 0              & 0             & 0                          \\ \hline
            I - Arithmetic                      & 1                     & 1                                   & 1                 & 01               & 0                 & 0              & 0             & 0                          \\ \hline
            I - Load                            & 1                     & 1                                   & 1                 & 01               & 1                 & 0              & 1             & 0                          \\ \hline
            SB                                  & 1                     & 0                                   & 0                 & 11               & 0                 & 0              & 0             & 1                          \\ \hline
            S                                   & 1                     & 0                                   & 1                 & 01               & 0                 & 1              & 0             & 0                          \\ \hline
        \end{tabular}
    }
\end{table}

\begin{table}[!h]
    \centering
    \caption{Control Signal Description}
    \label{table:control_description}
    \resizebox{0.8\textwidth}{!}{
        \begin{tabular}{|l|l|}
            \hline
            \multicolumn{1}{|c|}{\textbf{Control Signal}} & \multicolumn{1}{c|}{\textbf{Description}}                 \\ \hline
            load                                          & Program Counter Load Signal to advance PC                 \\ \hline
            register\_file\_rw                            & rw control signal for writing into Register File          \\ \hline
            alu\_src                                      & B data source for ALU                                     \\ \hline
            alu\_op                                       & ALU Operation                                             \\ \hline
            memtoreg                                      & To select which data is to be written into Register File  \\ \hline
            write                                         & write signal for Data Memory                              \\ \hline
            read                                          & read signal for Data Memory                               \\ \hline
            branch\_condition                             & to indicate the current instruction is branch instruction \\ \hline
        \end{tabular}
    }
\end{table}

\newpage
\section{Clock Divider}
Clock Divider is responsible for reducing the system clock to a desired CPU clock.
One of the reason that needs the Clock Divder is that, some module will only work on a specific clock frequency to ensure integraty.
\subsection{Implementation and Operation}
The source code for Clock Divider
can be accessed using \url{https://github.com/kaung-minkhant/risc-v-deca/blob/master/cpu/clk_divider.vhd}.
The overall block diagram can be seen in Figure \ref{block_diagram:cd}.
\insertBlockDiagram{cd_block_diagram}{1}{Block Diagram of Clock Divider}{block_diagram:cd}

\begin{table}[!h]
    \centering
    \caption{Input/Output of Clock Divider}
    \label{table:io_cd}
    \resizebox{0.4\textwidth}{!}{
        \begin{tabular}{|l|l|l|}
            \hline
            \multicolumn{1}{|c|}{\textbf{Signal}} & \multicolumn{1}{c|}{\textbf{Description}} & \multicolumn{1}{c|}{\textbf{Port}} \\ \hline
            clk                                   & system clock                              & Input                              \\ \hline
            reset\_n                              & system reset                              & Input                              \\ \hline
            clk\_o                                & CPU clock                                 & Output                             \\ \hline
        \end{tabular}
    }
\end{table}
\newpage
\paragraph*{Operation}
The Clock Divider works on a simple principle.
On every rising edge of the system clock, the internal counter is increment.
Once the counter reached the desired frequency count, the output simply needs to be toggled between High and Low, creating a clock of desired frequency.

\section{Pipelining}
Pipelining in CPU design is a technique used to enhance the efficiency and throughput of instruction processing.
It breaks down the execution of instructions into a series of stages, with each stage being handled by a different segment of the CPU.
This allows multiple instructions to be in various stages of execution simultaneously, overlapping and speeding up the overall processing of instructions.

The key advantage of pipelining is that it allows different instructions to be processed simultaneously, improving the CPU's overall throughput.
However, for pipelining to work effectively, the stages must be carefully synchronized,
and potential hazards such as data dependencies (where one instruction depends on the result of a previous one) must be managed to avoid incorrect results.
This will be tackled in a later section.
Additionally, pipeline stalls can occur if a stage has to wait for a previous stage to complete its work, which can reduce the performance gains.

\subsection{Why use Pipelining?}
Pipelining is used in CPU design for several important reasons, all aimed at improving the efficiency and performance of the central processing unit (CPU):
\paragraph*{Increased Throughput}
Pipelining allows multiple instructions to be in various stages of execution simultaneously.
As a result, the CPU can process a new instruction in each clock cycle, greatly increasing the overall throughput and execution speed of instructions.
\paragraph*{Reduced Latency}
By breaking down instruction processing into stages, pipelining reduces the time it takes to complete an individual instruction.
This reduction in latency means that the CPU can start processing the next instruction before the previous one is finished,
making better use of the available time.
\paragraph*{Resource Utilization}
Pipelining improves the utilization of CPU resources. While one stage of the pipeline is working on an instruction,
other stages can simultaneously work on different instructions.
This ensures that CPU components like the arithmetic logic unit (ALU) and registers are continuously engaged, maximizing their efficiency.
\paragraph*{Improved Performance Scaling}
As CPU clock speeds increase, it becomes challenging to increase performance solely by increasing clock frequency due to power and heat constraints.
Pipelining is one of the techniques used to improve performance without dramatically increasing clock speeds.
It allows designers to achieve more work per clock cycle.
\subsection{How Pipelining Works?}
Pipelining in CPU design breaks down the execution of instructions into several stages, allowing multiple instructions to be processed simultaneously.
Each stage of the pipeline performs a specific function, and as an instruction progresses through the pipeline, new instructions can enter the pipeline,
making more efficient use of the CPU's resources. Here's how pipelining works with its stages:
\paragraph*{Fetch Stage (IF - Instruction Fetch)}
In this stage, the CPU fetches the next instruction from Instruction Memory.
The program counter (PC) or instruction pointer points to the memory location of the instruction.
Simultaneously, the previous instruction is being decoded and executed in subsequent stages.
\paragraph*{Decode Stage (ID - Instruction Decode)}
In this stage, the fetched instruction is decoded.
The CPU determines the operation to be performed and identifies the registers or memory locations involved.
While this decoding occurs, the next instruction is fetched in the fetch stage.
\paragraph*{Execute Stage (EX - Execution)}
In this stage, the actual computation or operation specified by the instruction is carried out.
This can involve arithmetic calculations, logical operations, or data manipulation.
While execution is happening, the next instruction is being decoded, and the instruction after that is being fetched.
\paragraph*{Memory Stage (MEM - Memory Access)}
Not all CPUs have this stage, but in Microcontroller that interact frequently with Data Memory that connects to peripherals,
this stage handles memory-related operations like loading data from memory or storing data to memory.
Like the other stages, it operates simultaneously with the other stages, and the next instruction is in the decode stage while memory operations occur.
\paragraph*{Write Back Stage (WB - Write Back)}
In this final stage, the results of the executed instruction are written back to the appropriate registers in Register File.
At the same time, a new instruction is fetched in the fetch stage, starting the process again.

\newpage
\insertGraphic{pipeline_execution}{0.5}{0}{Sample Pipeline Execution with respect to CPU Clock}{graphic:pipeline_execution}
A sample pipeline execution would be as follows and can be viewed graphically in Figure \ref{graphic:pipeline_execution}:
\begin{itemize}
    \item In the first clock cycle (1), Instruction 1 is in the Fetch stage (IF), Instruction 2 is in the Decode stage (ID), and so on.
    \item In the second clock cycle (2), Instruction 1 moves to the Decode stage (ID), Instruction 2 moves to the Execute stage (EX), and Instruction 3 moves to the Decode stage (ID), and so on.
    \item This continues for subsequent clock cycles, with each instruction advancing through the stages, and new instructions being fetched.
\end{itemize}

\subsubsection{Performance Analysis}
If the following instruction delays are assumed in Figure \ref{graphic:sample_delay}
\insertGraphic{sample_delay}{0.4}{0}{Sample Instruction Delays}{graphic:sample_delay}

The following simple analysis can be done with respect to computation time, as can be seen in Figure \ref{graphic:performance_pipeline}
As can be observed in Figure \ref{graphic:performance_pipeline}, the non-pipeline cpu will take more execution time as the number of instructions grows.
Thus, for prgrams with large instrucitons, pipelining is necessary in order to increase the instruction throughput, other than increasing clock frequency.
\insertGraphic{performance_pipeline}{0.5}{0}{Sample Analysis on Computation Time}{graphic:performance_pipeline}

\newpage
\subsection{Pipelining Architecture}
\insertBlockDiagram{cpu_pipeline_block_diagram}{0.8}{Pipelined CPU Block Diagram with Pipeline Registers}{block_diagram:cpu_pipeline}
In Figure \ref{block_diagram:cpu_pipeline}, it can be seen that a simple pipeline structure is implemented in this work.
Between each stage, specific pipeline registers are inserted to latch in the previous results.
All pipeline registers are synchronized by CPU clock.
Description and Responsibilities of each pipeline registers will be described below.
\paragraph*{IF/ID Register}
This register is responsible for latching the current instruction. It will provide the instruction to the ID stage.

\paragraph*{ID/EX Register}
This register is responsible for keeping the contents of the accessed registers by the current instruction.
It also keeps current source register addresses, and destination register address, opcodes to make decisions late in the pipeline.
This will be of importance when dealing with pipeline hazards.

\paragraph*{EX/MEM Register}
This register is to keep all the results from ALU.
The data that needs to be written into Data Memory will be kept in this register.
It also keeps source register addresses, and destination register address, opcodes to make decisions late in the pipeline.

\paragraph*{MEM/WB Register}
This register holds data, that will be stored back into Register File.
It also latch in destination register address so that the CPU knows where to write the data to.

\subsection{Pipelining Hardwre}
The implementation of the above pipeline registers can be access through \url{https://github.com/kaung-minkhant/risc-v-deca/tree/master/cpu}.
The files are named based on the name of the register in Figure \ref{block_diagram:cpu_pipeline}.
All of these registers are implementated as simple registers, thus simplifying the hardware implementation.

\subsection{Pipelining Hazards}
Pipelining in CPU design, while highly effective at improving performance,
can encounter various hazards or issues that need to be carefully managed to ensure the correct execution of instructions
and to prevent performance degradation. The main types of pipelining hazards are:

\paragraph*{Structural Hazards}
This occurs when multiple pipeline stages need the same CPU resource simultaneously.
For example, if the Fetch stage and Memory stage both require access to the memory unit, a structural hazard arises.
This can be addressed through resource allocation and scheduling techniques.
In this work, this hazard will not occur, since all the hardware resources are seperated and will only be used by a single stage.
The instructions are designed to behave this way as well.

\paragraph*{Data Hazards}
\begin{itemize}
    \item \textbf{Read-After-Write (RAW) Hazard}: Also known as a data dependency hazard,
          this occurs when an instruction depends on the result of a previous instruction that is still in the pipeline.
          For example, if Instruction A writes to a register, and Instruction B reads from the same register,
          B must wait until A's result is available.
    \item \textbf{Write-After-Read (WAR) Hazard:} This happens when an instruction writes to a register that another instruction reads from,
          causing the reading instruction to get the wrong data. This hazard is less common than RAW hazards but can be resolved
          using proper data forwarding.
\end{itemize}

\paragraph*{Control Hazards}
When a branch instruction is encountered, the pipeline may have already fetched and partially processed subsequent instructions.
If the branch condition is resolved late, it can lead to wasted work and incorrect instruction execution.

\subsection{Data Harzards Mitigation}
Data Hazards can be mitigated through a process called Data Forwarding.
Data forwarding, in the context of a pipeline CPU, is like giving instructions a shortcut to quickly get the data they need to work correctly.
Since each instruction may need data from different stages of the pipeline, various forwarding paths are created.
Without data forwarding, the CPU would have to pause and wait for the result to be stored in memory before the second instruction can use it.
This pause would slow down the whole process. But with data forwarding,
the first instruction's result can be directly passed to the second instruction, allowing the CPU to work faster and more efficiently.

\subsubsection{Mitigation Hardware}
As stated above, all of the necessary data to determine data dependencies are latched into pipeline registers.
Thus, Data Forwarding Hardware can easily retrieve these data, make decision, and then forward the right data to the next instruciton.
The Data Forwarding Hardware is positioned in EXE stage, since most of the data are forwarded to the ALU, which is in EXE stage.
The Data Forwarding Hardware block diagram can be seen in Figure \ref{block_diagram:data_forwarding}.
Since the data are required for execution stage, and there are two inputs that are fed into the ALU, two forwarding outputs are required.
All the required data to make proper decision are included in the diagram, with the stage into which they are latched.
This hardware only generate the forwarding signal, and this signal control the multiplexer that select the correct data to feed the input of ALU.
\io{opcode\_i} is taken from ID stage.
\insertBlockDiagram{df_block_diagram}{0.7}{Block Diagram of Data Forwarding Unit}{block_diagram:data_forwarding}

\newpage
The implementation algorithm can be found on \url{https://github.com/kaung-minkhant/risc-v-deca/blob/master/cpu/forwarding_unit.vhd}.
The algorithm will be demonstrated through an example below.
\begin{center}
    addi x1 x2 x3\\
    addi x2 x1 x3\\
    addi x3 x1 x2
\end{center}

The first instruction, in the example does not have any dependencies since it is at the start of the program.
The second instruciton, however, depends on the result of first instruction and the first instruction is a write instruction.
Thus, forwarding data from the destination of the first instruction to the first source of second instruction is needed.
This condition is checked in the forwarding hardware and generate appropriate forwarding signal to forward the ALU result of the first instruction to the source of the second instruction.

If the third instruction is inspected, it depends on both the results of the first and second instructions.
Both of the first two instructions are write instructions, and thus, this condition is caught by the forwarding hardware.
This allows the hardware to generate the required signals to forward the ALU result of first instruction, which will be in WB stage, and to forward the ALU result of second instruction
which is in MEM stage. It should be noted that the hardware first check whether the previous instructions are write instructions, such as add, addi, loads, etc.

\subsubsection{Data Memory Design for Pipeline}
As described in the section of Data Memory, it is stated that the Memory is designed to facillitate the operation of pipeline execution.
Normally, Data Memory reads and writes on the rising edge of CPU clock.
However, all of the pipeline register latch data on the rising edge of the CPU clock. The problem can be seen in the following sample program.

\begin{center}
    ld x2 x0 0 \\
    add x1 x2 x3
\end{center}

In this program, the first instruction is a load. This means to load the content of data at location 0 of Data Memory.
The second instruction is an add instruction that depends on the result of the first instruciton.
When the second instruction is in EXE stage, where it needs the result of load instruction, the first instruction is still in MEM stage.
Since the Data Memory reads on the rising edge of the clock, the contents will not be available at the current clock cycle, which means the pipeline need to be stalled.
By allowing the Data Memory to be able to read on the falling edge of the cycle, the content will be availabe in the same clock to be forwarded to the second instruction.

\subsection{Control Hazards Mitigation}
Control Hazards are mitigated, in the same way as Data Hazards.
Unlike Data Hazard Mitigation, the forwarding hardware for control hazards, is positioned in ID stage.
In addition to it, the Branching Hardware, which acturally make the decision, is also placed in ID stage.
The reason is that the pipeline needs to know, whether to branch or not, before feeding data to EXE stage.
Without this prior decision, if the program need to branch, all of the data will need to be flushed out, thus requiring another hardware.
Thus by placing the forwarding hardware in ID stage and forwarding all the necessary data to it, the pipeline knows whether to branch or not, during Instruction Decoding stage.

\subsubsection{Mitigation Hardware}
The block digram describing input and output required for forwarding can be seen in Figure \ref{block_diagram:dfb}
\insertBlockDiagram{dfb_block_diagram}{0.7}{Block Diagram of Branch Forwarding Unit}{block_diagram:dfb}
As stated in Branching Hardware section, it only need the two pieces of data, then make decision whether to branch or not.
Forwarding Hardware makes sure to feed in the correct data. The algorithm is very similar to the one utilized in Data Forwarding Hardware.
The implementation can be view through \url{https://github.com/kaung-minkhant/risc-v-deca/blob/master/cpu/forwarding_unit_branch.vhd}.
Just as the Data Forwarding Hardware, this checks the destination address of the previous instructions, which will be in different stages, against
the source addresses of the branch instruction.
Just like the Data Forwarding Hardware, this forwarding Hardware generate the signal to control the multiplexers sitting at the inputs of Branching Hardware.

One distinction that sets this forwarding hardare apart from Data Forwarding Hardware is stall detection.
This will be explained using a sample example below.
\begin{center}
    ld x2 x0 0 \\
    beq x1 x2 0
\end{center}

The first instruction is a load, and the second instruction is a branch instruction which depends on the load instruction.
However, the result of the load can only be forwarded when the load instruction reaches the MEM stage.
As stated above, the branching decision is made in ID stage, meaning that the load is in EXE stage, thus, eninevitably resulting in a stall.
During a stall, all controll signals are cleared and PC is freezed, until the stall is lifted.

\newpage
\section{Connecting CPU to Peripheral}
Connecting CPU with Peripherals is accomplished by assigning specific memory locations to the peripherals, and the process is called port mapping.
There are two types of port mapping, input port mapping and output port mapping.
Input port mappings updates the respective Data Memory locations whenever they are updated.
They are the outputs of peripherals.
These mappings are read only, currently, by the CPU.
Output port mappings are Data Memory locations and the data is set by CPU and accessed by the peripherals.
These are inputs to Peripherals.

\subsection{Data Memory Mapping}
Data Memory is allocated with 8-bit address, thus allowing for 256 spaces.

\begin{table}[!h]
    \centering
    \caption{Data Memory Mapping}
    \label{table:memory_map}
    \begin{tabular}{|c|l||c|l|}
        \hline
        \multicolumn{1}{|l|}{Address} & Description           & \multicolumn{1}{l|}{Address} & Description           \\ \hline
        0                             & NULL                  & 20                           & spi1\_data\_write     \\ \hline
        1                             & general\_pin\_dir     & 21                           & spi1\_data\_read      \\ \hline
        2                             & general\_pin\_write   & 22                           & \multirow{2}{*}{NULL} \\ \cline{1-3}
        3                             & general\_pin\_read    & 25                           &                       \\ \hline
        4                             & uart1\_controls       & 26                           & i2c1\_data\_write     \\ \hline
        5                             & spi1\_controls        & 27                           & i2c1\_data\_read      \\ \hline
        6                             & i2c1\_controls        & 28                           & i2c1\_addr            \\ \hline
        7                             & \multirow{2}{*}{NULL} & 29                           & \multirow{2}{*}{NULL} \\ \cline{1-1} \cline{3-3}
        10                            &                       & 30                           &                       \\ \hline
        11                            & uart1\_data\_write    & 35                           & uart1\_flags          \\ \hline
        12                            & uart1\_data\_read     & 38                           & spi1\_flags           \\ \hline
        13                            & uart1\_data\_read\_32 & 37                           & i2c1\_flags           \\ \hline
        14                            & \multirow{2}{*}{NULL} & 38                           & \multirow{2}{*}{USER} \\ \cline{1-1} \cline{3-3}
        19                            &                       & 255                          &                       \\ \hline
    \end{tabular}
\end{table}

