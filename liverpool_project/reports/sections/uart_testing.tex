\subsection{Testings of UART Modules}
The first test for the UART module is performed using a data loop back test. This test requires the the transmitted data to be fed back to the receiving line, making a loop. The second test perform testing on detection of frame error for the receiving data. The follwing tests will be performed.

\begin{enumerate}
	\item Transmitting same data 2 times with 32 bit mode off
	\item Transmitting different data with 32 bit mode off
	\item Transmitting same data 2 times with 32 bit mode on
	\item Transmitting different data with 32 bit mode on
	\item Testing detection of frame error
	\item Baudrate Testing
\end{enumerate}

\subsubsection{Testbench Setup}
Testbench for first test includes the transmitter module, transmitting data on the tx line and the tx line is connected into the rx data line, creating a loop. 

\insertBlockDiagram{uart_module_tb}{0.5}{Block Diagram of testbench for UART Module Testing}{block_diagram:uart_module_testing}


\newpage
\subsubsection{Transmitting same data 2 times with 32 bit mode off}
In this loop back test, the module is set to transmit the same data 2 times, one after another. 32 bit receive mode is turned off (\io{rx\_mode\_32\_tb} = 0), thus, \io{rx\_done\_tb} is pulsed high after receiving every byte without frame error. In figure \ref{uart_transmit_same_twice}, it can be seen transmitting the correct data and the same data has been received without frame error. \io{rx\_done\_tb} pulsing pattern can also be seen. Little-endian effect can be seen clearly.

\insertWaveform{uart_transmit_same_data}{0.6}{Simulation waveform for transmitting the same data twice with 32 bit mode off}{uart_transmit_same_twice}

\subsubsection{Transmitting different data with 32 bit mode off}
In this loop back test, the module is set to transmit the two different data, one after another. 32 bit receive mode is turned off (\io{rx\_mode\_32\_tb} = 0), thus, \io{rx\_done\_tb} is pulsed high after receiving every byte without frame error. In figure \ref{uart_transmit_different}, it can be seen transmitting the correct data and the same data has been received without frame error. \io{rx\_done\_tb} pulsing pattern can also be seen. Little-endian effect can be seen clearly.

\insertWaveform{uart_transmit_different_data}{0.6}{Simulation waveform for transmitting different data with 32 bit mode off}{uart_transmit_different}

\subsubsection{Transmitting same data 2 times with 32 bit mode on}
In this loop back test, the module is set to transmit the same data 2 times, one after another. 32 bit receive mode is turned on (\io{rx\_mode\_32\_tb} = 1), thus, \io{rx\_done\_tb} is pulsed high only after receiving four bytes without frame error. In figure \ref{uart_transmit_same_twice_32}, it can be seen transmitting the correct data and the same data has been received without frame error. \io{rx\_done\_tb} pulsing pattern can also be seen. Little-endian effect can be seen clearly.

\insertWaveform{uart_transmit_same_data_32}{0.6}{Simulation waveform for transmitting the same data twice with 32 bit mode on}{uart_transmit_same_twice_32}

\subsubsection{Transmitting different data with 32 bit mode on}
In this loop back test, the module is set to transmit the two different data, one after another. 32 bit receive mode is turned on (\io{rx\_mode\_32\_tb} = 1), thus, \io{rx\_done\_tb} is pulsed high after receiving four bytes without frame error. In figure \ref{uart_transmit_different_32}, it can be seen transmitting the correct data and the same data has been received without frame error. \io{rx\_done\_tb} pulsing pattern can also be seen. Little-endian effect can be seen clearly.

\insertWaveform{uart_transmit_different_data_32}{0.6}{Simulation waveform for transmitting different data with 32 bit mode on}{uart_transmit_different_32}

\paragraph*{Usernote}
As can be seen in the simulation wave forms, in order to read in the received data, the falling edge of the \io{rx\_done\_tb} has to be used to avoid metastability.

\subsubsection{Testing detection of frame error}
In this test, only the receiver is utilized to test frame error detection. A series of bit data will be generated by the testbench, without frame error first, and then with the frame error. In figure \ref{frame_error_testing}, the first data packet is sent without frame error. The second one is sent without start bit and the third one is without stop bit. As can be seen, the module checks for each received frame and respond the error through \io{frame\_error}.

\insertWaveform{frame_error_testing}{0.6}{Frame Error Testing}{frame_error_testing}

\newpage
\subsubsection{Baudrate Testing}
In this test, baudrate testing is performed. Two frequencies are used for this UART Module. Input system clock frequency delivered on \io{clk} and the required baudrate set by the user when initializing the module. Note that baudrate must always be smaller than input system clock frequency. In this test only, input system frequency of $10$ MHz and baudrate of $5$ MHz is used. This will translate to transmitting each bit every two system clock cycles as can be seen in figure \ref{baudrate_testing}.

\insertWaveform{baudrate_testing}{0.6}{Baudrate testing at $10$ MHz system clock with $5$ MHz baudrate}{baudrate_testing}
